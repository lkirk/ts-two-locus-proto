\documentclass[12pt]{article}

\usepackage{hyperref}
\usepackage{minted}
\usepackage{cite}
\usepackage{amsmath}
\usepackage[bb=dsserif]{mathalpha}
\usepackage{svg}
\usepackage{relsize}


\title{Two Locus General Statistics in tskit}
\author{Lloyd Kirk}

\begin{document}

\maketitle

\section{Background}

The intention of this document is to describe the changes to tskit to provide a
generalized framework for computing two-locus statistics for branches and sites.

Much of the design proposal is taken from this
\href{https://github.com/tskit-dev/tskit/pull/432}{PR} There has been a bit of
discussion \href{https://github.com/tskit-dev/tskit/issues/1900}{here} as well.

\section{Goal/Scope}

Ultimately, we intend to deprecate the LD calculator that is currently
implemented in tskit and implement a general framework for two-locus
statistics. This framework will create an interface for implementing new summary
statistics and implement the following features:

\begin{itemize}
  \item Sites with more than 1 mutation (multiallelic sites)
  \item Polarization, where applicable
  \item Branch Statistics
  \item LD Stats beyond $r^{2}$
\end{itemize}

\section{Implementation}
We have been trying to follow the general design pattern laid out in the C
api. Most of the initial design documentation will revolve around creating the
necessary C interfaces to consume from the python api.

Some of this documentation is based on my (LK’s) interpretation of the design
patterns in the existing code. I have no insight into how things might change
apart from the scattered TODOs in the code. In short, please feel free to
correct my assumptions/understandings.

\subsection{Outer Layer}

We propose to add more functionality to
\mintinline{C}{tsk_treeseq_general_stat}, which is the main entrypoint for
computing stats from tree sequences. Within
\mintinline{C}{tsk_treeseq_general_stat}, we will add some more conditionals in
the form of \mintinline{C}{tsk_flags_t}.

\begin{minted}{c}
  bool stat_two_site = !!(options & TSK_STAT_TWO_SITE);
  bool stat_branch = !!(options & TSK_STAT_BRANCH);
\end{minted}

These flags would dispatch our entrypoints for computing two site/branch statistics.


\begin{minted}{c}
static int
tsk_treeseq_two_branch_general_stat(
	const tsk_treeseq_t *self,
	tsk_size_t state_dim,
	const double *sample_weights,
	tsk_size_t result_dim,
	general_stat_func_t *f,
	void *f_params,
	tsk_size_t num_windows,
	const double *windows,
	tsk_flags_t options,
	double *result
)
\end{minted}


\section{Two Site Statistics}

We would provide a similar interface to computing
\mintinline{C}{site_general_stats}, but the tree traversal algorithm will
differ, providing haplotype counts instead of site counts.

\begin{minted}{c}
static int
tsk_treeseq_two_site_general_stat(
	const tsk_treeseq_t *self,
	tsk_size_t state_dim,
	const double *sample_weights,
	tsk_size_t result_dim,
	general_stat_func_t *f,
	void *f_params,
	tsk_size_t num_windows,
	const double *windows,
	tsk_flags_t options,
	double *result
)
\end{minted}

\begin{minted}{c}
static int
compute_general_two_stat_site_result(
	tsk_site_t *site_a,
	tsk_site_t *site_b,
	double *state,
	tsk_size_t state_dim,
	tsk_size_t result_dim,
	general_stat_func_t *f,
	void *f_params,
	double *total_weight,
	bool polarised,
	double *result,
)
\end{minted}

\section{Branch Statistics}

\section{Summary Functions}

\begin{tabular}{llll} Statistic & Polarization & Normalization & Equation\\

\hline $D$ & Polarized & Total & $D = f_{ab} - f_{a}f_{b}$ \\

$D^{\prime}$ &
Polarized & Haplotype Weighted & $D^{\prime} = \frac{D}{D_{max}}$ \\

$D^{2}$ &
Unpolarized & Total & $D^{2} = D^{2}$ \\

$D_{z}$ & Unpolarized & Total & $D_{z}
= D (1 - 2 f_{a})(1-2f_{b})$ \\

$\pi_{2}$ & Unpolarized & Total & $\pi_{2} =
f_{a}f_{b}(1-f_{a})(1-f_{b})$ \\

$r$ & Polarized & Haplotype Weighted & $r =
\frac{D}{\sqrt{f_{a}f_{b}(1-f_{a})(1-f_{b})}}$ \\

$r^{2}$ & Unpolarized &
Haplotype Weighted & $r^{2} = \frac{D^{2}}{f_{a}f_{b}(1-f_{a})(1-f_{b})}$ \\
\end{tabular}
\\ 
\\
Where $D_{max}$ is defined as:

\[
  D_{max} = 
  \begin{cases}
    \min\{f_{a}(1-f_{b}),f_{b}(1-f_{b})\} & \text{if~}D>=0 \\
    \min\{f_{a}f_{b},(1-f_{b})(1-f_{b})\} & \text{otherwise}
  \end{cases}
\]

\subsection{Summary Function Signature}
Two locus statistics need to know the number of AB, Ab, and aB haplotypes. They
also need to know the total number of haplotypes being considered in order to
properly convert the counts of each haplotype to proportions.
\begin{minted}{c}
static int
summary_func(int w_AB, int w_Ab, int w_aB, int n)
\end{minted}

\begin{minted}{c}
two_site_summary_func(
	tsk_size_t state_dim,
	const double *state,
	tsk_size_t TSK_UNUSED(result_dim),
	double *result,
	void *params
)
\end{minted}

\subsection{Normalization}
In our testing of summary functions, we found that the appropriate normalization
procedure can vary depending on the summary function. We've settled on two
normalization procedures: ``Haplotype Weighted'' and ``Total''.

\subsubsection{Hapltype Weighted}
\[
  \sum_{i=1}^{n}\sum_{j=1}^{m}p(A_{i}B_{j})F_{ij}
\]
where $F$ is the summary function and $p(A_{i}B_{j})$ is the frequency of
haplotype $A_{i}B_{j}$. This method can be found
in~\cite{zhao2007evaluation}. We apply this 
\subsubsection{Total}
In the ``Total'' normalization method, we simply divide by the number of
haplotypes that we've visited. If we're
\[
  \frac{1}{(n-\mathbb{1}_{p}) (m-\mathbb{1}_{p})}\sum_{i=1}^{n}\sum_{j=1}^{m}F_{ij}
\]
where $\mathbb{1}_{p}$ is an indicator function conditioned on whether or not
our statistic is polarized.

\subsection{Evaluation}
To ensure the correctness of our implementation, we have devised a number of
test scenarios that will produce data at the theoretical limits of the
statistics we've implemented.

\subsubsection{Test cases}
Table

\begin{table}
  \begin{tabular}{lc} Name & $\left(\begin{array}{cc} Site A \\ Site B
                                     \\ \end{array}\right)$ \\

    \hline
    Correlated & $\left(\begin{array}{ccccccccc}
                          0 & 1 & 1 & 0 & 2 & 2 & 1 & 0 & 1 \\
                          1 & 2 & 2 & 1 & 0 & 0 & 2 & 1 & 2 \\
                        \end{array}\right)$ \\
    Uncorrelated & $\left(\begin{array}{ccccccccc}
                            0 & 0 & 0 & 1 & 1 & 1 & 2 & 2 & 2 \\
                            0 & 1 & 2 & 0 & 1 & 2 & 0 & 1 & 2 \\
                          \end{array}\right)$ \\
    Correlated Biallelic & $\left(\begin{array}{cccccccc}
                                    0 & 0 & 0 & 0 & 1 & 1 & 1 & 1 \\
                                    0 & 0 & 0 & 0 & 1 & 1 & 1 & 1 \\
                                  \end{array}\right)$ \\
    Uncorrelated Biallelic & $\left(\begin{array}{cccccccc}
                                      0 & 0 & 0 & 0 & 1 & 1 & 1 & 1 \\
                                      1 & 1 & 0 & 0 & 0 & 0 & 1 & 1 \\
                                    \end{array}\right)$ \\
    Repulsion Biallelic & $\left(\begin{array}{cccccccc}
                                   0 & 0 & 0 & 0 & 1 & 1 & 1 & 1 \\
                                   1 & 1 & 1 & 1 & 0 & 0 & 0 & 0 \\
                                 \end{array}\right)$ \\
  \end{tabular}
  \caption{
    \textbf{Test cases for validating statistics.}
    In each case, we have an A and B site, representing the two sites under
    consideration for computaiton of our statistics.
  }
  \label{table:test_cases}
\end{table}

\subsubsection{Polarized}
We'll begin with the polarized statistics. $D$ and $D^{\prime}$ sum to zero when
they are unpolarized, so we are only providing a polarized method to compute
them. We can test the correctness of $D$ by computing $D$ with two alleles that
are in full repulsion:

\[
\]

\subsubsection{Notation}

In the following section, we use pairs of 1-d row vectors to describe the
terminal allelic state in a given pair of trees. For example, if we're given a
tree like the one shown in Figure~\ref{fig:example_tree}.

\begin{figure}
    \centering
    % \includesvg[width=\textwidth]{figure1.svg}
    % \includesvg[width=\linewidth,inkscapelatex=false,pretex=\relscale{0.4}]{figure1.svg}
    \includesvg[width=\linewidth,inkscapelatex=false]{figure1.svg}
    \caption{
      \textbf{An example of a tree sequence used to compute two-locus statistics.}
      The left tree has one site with four states and the right tree has one
      site with three states.
    }
\label{fig:example_tree}
\end{figure}

In this example, the state matrix is a 2$\times$8 matrix, representing the
terminal state of each sample in the tree. For Figure~\ref{fig:example_tree},
our state matrix looks like:

\[
  \left(
    \begin{array}{cccccccc}
      2 & 2 & 3 & 3 & 1 & 1 & 1 & 1 \\
      1 & 1 & 1 & 1 & 2 & 2 & 2 & 0 \\
    \end{array}
  \right)
\]

\bibliographystyle{plain}
\bibliography{references}

\end{document}